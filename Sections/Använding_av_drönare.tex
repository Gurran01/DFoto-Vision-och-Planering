\section{Användning av drönare}
\label{Användning_av_drönare}
Detta avsnitt beskriver de principer som bör efterföljas vid användning av drönare.

\subsection{Terminologi}
\begin{itemize}
    \item \textbf{Icke medverkande människa} \\
    Person som inte aktivt godkänt användningen av drönare och dess medförda risker, och/eller ej deltar i ett event där dessa risker tydligt informerats. 
    \item \textbf{Fjärrpilot} \\
    Den som flyger drönaren är en så kallad fjärrpilot (drönarpilot).    
    \item \textbf{Operatör} \\
    Vid all flygning med drönare måste det finnas en ansvarig operatör, en så kallad drönaroperatör. Operatören (person) ansvarar för att flygningen genomförs på ett säkert sätt och att fjärrpiloten som utför flygningen har rätt kompetens. Operatören och fjärrpiloten kan vara samma person. 
    
\end{itemize}

\subsection{Utrustning}
DFoto innehar:
\begin{itemize}
    \item \textit{Dji Mavic Air} (första generationen, utgiven 2018). \\
    Drönaren har genom åren kraschat och gått sönder, samt reparerats av Gustav ''Gusa'' Sandell år 2022. I nuläget fungerar drönaren, men hittills har IMU-modulen (gyroskop och accelerometrar) bytts ut och ett batteri gått sönder. Drönaren har därmed genomgått en del slitage och det finns anledningar att införskaffa en ny, vidare om detta i avsnitt \ref{Framtida användning}. 
    
\end{itemize}


\subsection{Utbildning}
Inom de senaste åren (2020-2023) har Sveriges lagar blivit hårdare och ställer högre krav vid användning av drönare. Detta har gjort det svårare att få använda drönare, speciellt på offentliga platser och i städer där det finns icke medverkande människor. Det har hittills inte fastställts interna regler inom Chalmers område, därav bör användningen av drönare utgå från nationella regler och \textit{Operatörens} uppfattning av flygsituationen på Chalmers campus. 

Baserat på statistiken av drönarkrascher som orsakats av okunskap inom DFoto, och kostnaden för nya drönare, ska \textbf{Drönaranvsarige} (se avsnitt \ref{Rollerbeskrivningar} för definition) vara utbildad och inneha ett drönarkörkort. Drönarkörkortet ska vara av typen A1/A3, där A2 också är rekommenderat. Utbildningen ska bekostas av Datateknologsektionen givet att utbildningen har en rimlig kostnad. (Kostnaden per person uppgick år 2023 till 390kr.)

\subsection{Framtida användning}
\label{Framtida användning}
Drönarmodellen \textit{DJI Mavic Air} har enligt Dji nått sin EOL (End of Lifecycle) och Dji stödjer inte längre modellen med uppdateringar eller reservdelar. Reservdelar går dock att köpa på andra ställen som Ebay och Amazon, dock till överpris. 