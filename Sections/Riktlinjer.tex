\section{Riktlinjer}
Detta avsnitt beskriver DFotos riktlinjer.

\subsection{Alkoholpolicy}
Vid arbete som fotograf för DFoto ska man ha uppsikt och värna om utrustningen kommittén innehar. Därav får ej fotografen vara alkoholpåverkad i den grad att utrustningen riskeras skadas.  

\subsection{Personskydds- och integritetspolicy}
Vid fotografering och filmskapande blir individer på motiv i riskzon för inskränkning, speciellt på evenemang som involverar alkohol. För att undvika inskränkningar och eventuell avpublicering bör följande principer följas: 
\begin{enumerate}
    \item I situationer där individer kan känna aggression eller rädsla för en kamera, bör DFoto-fotografer ta avstånd från situationen direkt. 
    \item Material som kan inskränka eller potentiellt skada individers integritet och/eller säkerhet får ej innehas eller publiceras. 
    \item Utanför Chalmers campus måste fotografering anpassas efter situation. På allmän plats är det tillåtet att fotografera, men oftast är det bäst att fråga om tillstånd om \textit{icke medverkande människor} deltar på motiv. Räddningspersonal, väktare och ordningsvakter bör principiellt ej fotograferas. 
    \item Om en individ finns på material och ej vill att materialet ska vara publicerat eller existera, ska materialet avpubliceras och/eller raderas direkt utan krav på anledning. Detta är givet enbart då individen är identifierbar på materialet. 
\end{enumerate}

\subsection{Upphovsrätt och ägandeskap}
Allt material som DFoto skapar ägs av DFoto, och i förlängningen Datateknologsektionen och alla dess medlemmar. Då material används av Datateknologsektionens medlemmar, bör materialet refereras till DFoto (exempelvis tagga @dfoto\_chalmers på Instagram). Vid icke tillåtten användning av material från externa parter utanför sektionen bör anmälan göras, speciellt om användningen är i kommersiellt bruk. Sittande i DFoto har rätt att få innehåll som skapats av DFoto, avpublicerat och/eller raderat oavsett innehavare.  