\section{Arbetsätt}
Detta avsnitt beskriver hur DFoto ska arbeta gentemot olika parter. Arbetssätten gäller om inget annat är bestämt i förväg och/eller specificerat i avsnitt \ref{Specificerade uppdrag}. Arbete utanför Datateknologsektionen är tillåtet, men bör ej överväga eller påverka ordinarie arbetet  Prisrekomendation för externa artbeten finns under avsnitt \ref{Prisrekomendation}.

%Prisrekomenation- och kalkyl finns i bilaga ???
\subsection{Generell arbetsprocedur} 
DFotos grundläggande procedur för foto- och/eller filmuppdrag är:
\begin{enumerate}
    \item Extern kommunikation och bokning av uppdrag för fotografering och filmning. 
    \item Intern samordning av arbetsschema och -fördelning. 
    \item Fotografera och/eller filma uppdragsevent.
    \item Efterbehandling av foto- och/eller filminnehåll.
    \item Publicering av foto- och/eller filminnebåll. 
\end{enumerate}

\subsection{Arbetsätt inom Datateknologsektionen}
\label{Arbetsätt inom Datateknologsektionen}
Inom Datateknologsektionen arbetar DFoto för att dokumentera sektionens studentkulturshistora, därav bör DFoto utföra fotouppdrag på sektionens betydelsefulla event. 
DFoto är ej ålagd att utföra uppdrag som ej kommunicerats minst en vecka i förväg och/eller ej godkänts av relevanta sittande inom DFoto. 

För uppdrag med platsbegränsning ska DFoto aktivt kommunicera hur många och vilka som ska vara fotografer minst tre dagar innan uppdraget. Det ska finnas biljetter för två fotografer för sådana event. 

DFotos fotografer är arrangörer på alla uppdrag som kommittén utför, därav ska alla nödvändiga kostnader för att närvara och arbeta på uppdrag täckas av värdarna för uppdraget. Detta är givet om inget annat är kommunicerat och överenskommet i förväg med uppdragets värdar och relevanta sittande i DFoto. Fotograferna har en vecka ifrån uppdragets slutdatum på sig att leverera fotoinnehåll om inget annat är överenskommet. 

\subsection{Arbetssätt inom Chalmers}
Nedan beskrivs standardarbetssätten för olika externa parter inom Chalmers. 

\subsubsection{Arbetssätt mot andra sektioner}
Mot andra sektioner hanteras och godkänns uppdrag individuellt, det vill säga kan externa uppdrag ej bli prenumererade.  

För uppdrag med platsbegränsning ska DFoto aktivt kommunicera hur många och vilka som ska vara fotografer minst tre dagar innan uppdraget. DFoto avgör hur många fotografer som behövs för att utföra uppdraget med rimlig arbetsbelastning. 

DFotos fotografer är medarrangörer på alla uppdrag som kommittén utför, därav ska alla nödvändiga kostnader för att närvara och arbeta på uppdrag täckas av värdarna för uppdraget. Detta är givet om inget annat är kommunicerat och överenskommet i förväg med uppdragets värdar och relevanta sittande i DFoto. Eventuell kompensation för uppdrag bör kommuniceras och vara bestämt innan godkännande av uppdrag (prisrekomendation finns under avsnitt \ref{Prisrekomendation}). Fotograferna har en vecka ifrån uppdragets slutdatum på sig att leverera fotoinnehåll om inget annat är överenskommet. 

\subsubsection{Arbetssätt mot kåren och kårkommittéer}
\label{Arbetssätt - Arbetssätt mot kåren och kårkommittéer}
Mot kåren och kårkommittéer hanteras och godkänns uppdrag individuellt, det vill säga kan externa uppdrag ej bli prenumererade.  

För uppdrag med platsbegränsning ska DFoto aktivt kommunicera hur många och vilka som ska vara fotografer minst en vecka innan uppdraget. DFoto avgör hur många fotografer som behövs för att utföra uppdraget med rimlig arbetsbelastning, givet att inget är explicit kommunicerat i uppdraget. 

DFotos fotografer är arrangörer på alla uppdrag som kommittén utför, därav ska alla nödvändiga kostnader för att närvara och arbeta på uppdrag täckas av värdarna för uppdraget. Detta är givet om inget annat är kommunicerat och överenskommet i förväg med uppdragets värdar och relevanta sittande i DFoto. Uppdraget ska kompenseras enligt gällande prisrekomendation enligt avsnitt \ref{Prisrekomendation}). Fotograferna har en vecka ifrån uppdragets slutdatum på sig att leverera fotoinnehåll om inget annat är överenskommet. Levererat innehåll ska vara vattenstämplat med DFotos officiella logga, givet att inget annat är överenskommet. 

\subsubsection{Arbetssätt mot Chalmers Tekniska Högskola}
Detsamma som för \ref{Arbetssätt - Arbetssätt mot kåren och kårkommittéer} Arbetssätt mot kåren och kårkommittéer. 

\subsection{Arbetsätt mot externa parter}
Nedan beskrivs standardarbetssättet för externa parter utanför Chalmers. 

\subsection{Prisrekomendation}
\label{Prisrekomendation}
Prisrekomendationen finns som bilaga under avsnitt \ref{Bilaga - Prisrekomendation}.  Innan betalda uppdrag kan godkännas av DFoto, bör de ha godkänts av D-styret. Fakturering utförs av Datateknologsektionens sektionskassör. 
