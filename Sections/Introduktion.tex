\section{Introduktion}    
Rapporten beskriver de grundläggande principer, arbetssätt, policys och regler som tagits fram av DFoto'22 med bidrag från DFoto'21 och D-Styret'22. Rapporten fungerar som ett överlämningsmaterial mellan Datateknologsektionens generationer av DFoto, men även som en informativ grund för övriga kommittéer och styren.



%Informationen och principerna i rapporten är ej skriv i sten, utan kan förändras, men bör vara som utgångspunkt för påstigande generationer av DFoto och bör enbart ändras efter samråd med styren och DFoto-pateter.

\subsection{Syfte}
Rapporten syftar till att ge en informativ grund för hur DFotos verksambet ska bedrivas och utvecklas, för att öka och säkerhetställa kvalitén på DFotos arbete och resultat. Rapporten syftar till att standardisera DFotos arbetssätt och därigenom förtydliga och förenkla kommitténs arbete. Därav syftar rapporten till att minimera problem, oklarheter, stress och fel uppkomna utav okunskap. 

\subsection{Mål}
Målet med rapporten är att sittande i DFoto ska jobba i enlighet med framtagna standarder och ha vetskap om vilka principer som är framtagna, samt finna trygghet i att information finns att tillgå i rapporten vid eventuella oklarheter. 

\subsection{Terminologi}
Förklaringar till ord och förkortningar som används i rapporten.

\begin{itemize}
    \item \textbf{Uppdrag} \\
    Ett uppdrag är ett ärende/uppgift för kommittén. Exempelvis är att fotografera en gasque är ett uppdrag. 
    \item \textbf{Prenumererade uppdrag} \\
    Uppdrag som DFoto per automatik är ålagda att utföra, om inget annat har kommunicerats i förväg. 
    \item \textbf{Medverkande människor} \\
    Människor som medvetet och villigt deltar/är identifierbara på foto-/filminnehåll
    \item \textbf{Icke medverkande människor} \\
    Människor som omedvetet deltar och/eller är identifierbara på foto-/filminnehåll. 

\end{itemize}