\section{Specificerade uppdrag}
\label{Specificerade uppdrag}
I detta avsnitt beskrivs återkomande uppdrag och deras överenskomna standardutförande. \textit{Prenumererade} event är uppdrag som DFoto per automatik är ålagda att utföra, om inget annat har kommunicerats i förväg. 

\subsection{DNollK}
Datateknologsektionens mottagningskommittés uppdrag:
\begin{itemize}
    \item \textbf{Portträtt till Nollmodulen} (Maj-Juni)\\
    På uppdrag från DNollK arrangerar DFoto porträttfotografering för datateknologsektionens kommittéer. DFoto är skyldiga att kontakta och kommunicera arrangemanget till berörda kommittéer. Om kommittéer ej avtalat tid för porträttfotografering är DFoto ej ålagda att utföra uppdraget eller levera innehåll. (Ett tips är att informera alla kommittéer om upplägget och boka in dem så snart som möjligt. En porträtt-kalender hade varit en möjlighet.)
    
    \item \textbf{Mottagningen} (Augusti-September)\\
    Mottagningen involverar alla officiella och öppna event som arrangeras av datateknologsektionens kommittéer under Chalmers mottagningsveckor. Dessa event är prenumererade. 
    
\end{itemize}

\subsection{DMNollK}
Datateknologsektionens Master-mottagningskommittés uppdrag:
\begin{itemize}

    \item \textbf{Master-Mottagningen} (2 veckor i September)\\
    Master-Mottagningen involverar alla officiella och öppna event som arrangeras av datateknologsektionens kommittéer i samband med Master-mottagningens mottagningsveckor. Dessa event är prenumererade. 
    
\end{itemize}

\subsection{D6}
Datateknologsektionens Sexmästeris uppdrag:
\begin{itemize}
    \item \textbf{Gasque} \\
    Gasque är ett prenumererat event som följer DFotos generella arbetsprocedur för Datateknologsektionen. En gasque omfattar både sittning och eftersläpp. 
    \item \textbf{Julbord} \\
    Julbordet är subventionerat, men kostade för DFoto'22. Skulle rekommendera att diskutera med D6 en lösning för Julbord framöver. Hade varit uppskattat med en fotovägg med jultema.  
    \item \textbf{MTS} (Februari)\\
    Privat finsittning i Gasquen med sexmästerier och GasqueK. Eventet involverar att fota sittningen parallellt med en fotovägg. Fotoväggen utförs likt CV-fotografering (se avsnitt \ref{Specificerade uppdrag - DAG} DAG - CV-fotografering) med en bakgrund med D6-tema. Arbetsmat och eventuell överenskommen kompensation ska täckas av D6 då detta är ett privat event som ej faller under DFotos normala ålägganden gentemot Datateknologsektionen. 
\end{itemize}

\subsection{Delta}
Datateknologsektionens PR-kommittés uppdrag: 
\begin{itemize}
    \item \textbf{Pub}\\
    Pub är ett prenumererat event i form av tentapubar (CMP) och öppen pub under pubrundor. Fotografer bör vara på plats från öppningen för att få garanterat inträde, dock är det inte obligatoriskt att närvara till stängning. Arbetsmat täcks av Delta i form av en Deltaburgare per fotograf.  
    
\end{itemize}

\subsection{DAG}
\label{Specificerade uppdrag - DAG}
Datateknologsektionens ArbetsmarknadsGrupps uppdrag:
\begin{itemize}
    \item \textbf{CV-fotografi} (2-4 veckor innan DatE-IT mässan; NC-Bibblan)\\
    Porträttfotograffering med DATE-IT:s studiolampor och DAG:s projektorduk som bakgrund. DatE-IT har en bakgrundsduk och upphängning, men kräver mer arbete då duken är skrynklig. Använd ett stativ för att minimera 'motion blur', och fotografera på ca 5m avstånd med lång brännvid (mycket inzoomat) för att enbart personen och den vita bakgrunden ska synas i bilden. 
    \item \textbf{Företags-Pub}
    Pub arrangerat av DAG men sponsrat av ett utomstående företag. Eventet följer standardarbetsättet gentemot Datateknologsektionen för biljetter. 
    
\end{itemize}

\subsection{DatE-IT}
Arbetsmarknadsgruppen mellan Data, Elektro och IT som arrangerar Date-IT mässan och banketten. Uppdragen mot DatE-IT är under en dag och kräver sex fotografer totalt för att utföra alla uppdragen. Klädseln bör vara anständig och professionell (helst inte orangea byxor av erfarenhet).
\begin{itemize}
    \item \textbf{Mässa} (November)\\
    Allmän fotografering av montrar och närvarande studenter. 
    2 fotografer rekommenderas.
    \item \textbf{CV-fotografi} (Parallellt med Mässa; Konferensrum vid Betongsalen)\\
    Samma arbetsätt som för avsnitt \ref{Specificerade uppdrag - DAG} DAG - CV-fotografi. 2-3 fotografer rekommenderas.
    \item \textbf{Bankett} (Kväll efter mässa; Kårrestaurangen)\\
    Baketten är en finsittning. 2 fotografer rekommenderas. 
\end{itemize}

